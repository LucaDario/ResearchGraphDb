\section{Manuale del PoC}
\subsection{Introduzione}
In questa sezione verranno descritte come usare le varie librerie disponibili, per analizzare i vari contratti dei metodi dell'interfaccia comune \textit{ReputationController} vedere documentazione contenuta direttamente nella classe.




\subsection{Uso della libreria utilizzando OrientDB e/o Neo4j}
Con il seguente codice viene creata un applicazione Spring Boot che potr\'a sfruttare, grazie al campo dati \textit{reputationController} che implementa \textit{ReputationController}, i metodi d'interfaccia. Il corpo del metodo run, o in generale di qualsiasi metodo evocato dell'interfaccia, \'e indipendente dal tipo di implementazione scelta.
\begin{lstlisting}
package it.ldario;

import it.ldario.graphdbneo4j.ReputationControllerNeo4j;
//import it.ldario.graphdborientDb.ReputationControllerOrientDb;
import org.springframework.beans.factory.annotation.Autowired;
import org.springframework.boot.CommandLineRunner;
import org.springframework.boot.SpringApplication;
import org.springframework.boot.autoconfigure.SpringBootApplication;

@SpringBootApplication
public class MiaClasse implements CommandLineRunner {

    public static void main(String[] args) {
        SpringApplication.run(MiaClasse.class, args);
    }
	//esplicitando il tipo verr\'a scelto, rispettivaemnte, il reputation controller di Neo4j o OrientDB
    private ReputationControllerNeo4j reputationController;
    //private ReputationControllerOrientDb reputationController;


    @Autowired
    public MiaClasse(ReputationControllerNeo4j reputationController) {
        this.reputationController = reputationController;
    }

    @Override
    public void run(String... strings) throws Exception {
        //aggiunge al database scelto(orientDB oppure Neo4j), dipende dal tipo che viene dichiarato, un entity con id Luca
        reputationController.addAccountId("Luca");

        //stampa la reputazione totale dell'entity Luca
        System.out.println(reputationController.getTotalReputation("Luca"));

        //stampa il totale delle transazioni effettuale da Luca negli ultimi 15 giorni

        System.out.println(reputationController.getAmountAccountIdInATimeRange("Luca", 15));
        //aggiunge al database scelto(orientDB oppure Neo4j), dipende dal tipo che viene dichiarato, un account con id IT123
         TransactionBuilder transactionBuilder = new TransactionBuilder();
        transactionBuilder.setCountry("Italy").setCity("San Michele delle Badesse").setAmount(100).setCurrency("Euro");
        reputationController.addTransaction("Luca","IT123",transactionBuilder.build());
        

    }
}
\end{lstlisting}


\subsection{Uso di graphdb-insert}
Con il seguente codice vengono generati ed aggiunti al database scelto, decidendo il tipo del capo dati \textit{reputationController}, 10 entity, 10 account(i 10 account hanno rispettivamente un loro entity deciso in modo casuale) e 10 transazioni con payer, payee e dati casuali.
\begin{lstlisting}
package it.ldario;


import it.ldario.graphdbinsert.GraphDbInsertImpl;
import it.ldario.graphdbneo4j.ReputationControllerNeo4j;
//import it.ldario.graphdborientDb.ReputationControllerOrientDb;
import org.springframework.beans.factory.annotation.Autowired;
import org.springframework.boot.CommandLineRunner;
import org.springframework.boot.SpringApplication;


public class MiaClasse implements CommandLineRunner {

    //esplicitando il tipo verr\'a scelto, rispettivaemnte, il reputation controller di Neo4j o OrientDB
    private ReputationControllerNeo4j reputationController;
    //private ReputationControllerOrientDb reputationController;

    public static void main(String[] args) {
        SpringApplication.run(GraphdbPocApplication.class, args);
    }

    @Autowired
    public MiaClasse(ReputationControllerNeo4j reputationController) {
        this.reputationController = reputationController;
    }

    @Override
    public void run(String... strings) throws Exception {
        GraphDbInsertImpl graphDbInsert = new GraphDbInsertImpl(10,10,10,reputationController);

    }
}
\end{lstlisting}

\subsection{Uso di graph-csv}
Con il seguente vengono cretai 3 file csv nella root del progetto, uno per le entity, uno per gli account, uno per la relazione di appartenenza tra entity e account e un altro per le transazioni. \textbf{Questo \'e il miglior modo per creare un numero elevato di dati e poi inserirli nel database scelto tramite il suo importatore csv}.

\begin{lstlisting}
package it.ldario;


import it.ldario.graphdbcsv.GeneratorCsv;
import it.ldario.graphdbcsv.GeneratorCsvRandom;
import it.ldario.graphdbneo4j.ReputationControllerNeo4j;
import org.springframework.boot.CommandLineRunner;
import org.springframework.boot.SpringApplication;

//import it.ldario.graphdborientDb.ReputationControllerOrientDb;


public class MiaClasse implements CommandLineRunner {

    //esplicitando il tipo verr\'a scelto, rispettivaemnte, il reputation controller di Neo4j o OrientDB
    private ReputationControllerNeo4j reputationController;
    //private ReputationControllerOrientDb reputationController;

    public static void main(String[] args) {
        SpringApplication.run(GraphdbPocApplication.class, args);
    }

    @Override
    public void run(String... strings) throws Exception {
        GeneratorCsv generatorCsv = new GeneratorCsvRandom(100, 100, 300);
    }
}
\end{lstlisting}

\subsubsection{Importare il csv creato su Neo4j}
Seguire i seguenti passi in ordine:
\begin{itemize}
\item{1:} Inserire tutti i csv creati nella cartella destinata all'import di Neo4j\\ \url{https://neo4j.com/developer/guide-import-csv/}{https://neo4j.com/developer/guide-import-csv/}.
\item{2:} Eseguire in ordine e singolarmente i comandi contenuti in \textit{configurationCsv/neo4j/comandi.txt}
\end{itemize}

\subsubsection{Importare il csv creato su OrientDB}
Seguire i seguenti passi in ordine:
\begin{itemize}
\item{1:} Inserire sia csv appena creati che tutte le configurazioni json in \textit{configurationCsv/orientdb/*.json} in una cartella visibile a orientdb.
\item{2:} Eseguire tutti i file di configurazione json come descritto nel punto 3 al seguente indirizzo \url{https://orientdb.com/docs/2.2/Import-from-CSV-to-a-Graph.html}{https://orientdb.com/docs/2.2/Import-from-CSV-to-a-Graph.html} e nel seguente ordine: entity,account,entity\textunderscore account, transaction.
\end{itemize}






